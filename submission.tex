\documentclass{article}

\usepackage[english]{babel}
\usepackage[utf8]{inputenc}
\usepackage{amsmath,amssymb}
\usepackage{tabularx}
\usepackage{booktabs}
\usepackage{enumitem}
\usepackage{parskip}
\usepackage{graphicx}

\usepackage[top=2.5cm, left=3cm, right=3cm, bottom=4.0cm]{geometry}

\newcommand{\lectureheader}[4]{%
  \begin{minipage}{.3\textwidth}%
    
    \strut\includegraphics[width=\textwidth]{figures/ethlogo.pdf}%
  \end{minipage} \hfill%
  \raisebox{1.5mm}{%
    \begin{minipage}{0.69\textwidth}\sf\flushright%
        \textbf{\Huge #3}\mbox{\hspace{2mm}}\\#4\mbox{\hspace{2mm}}%
    \end{minipage}%
  }\\[-2mm]\hrule%
  \begin{minipage}[t]{0.5\textwidth}\sf\textit{#1} \end{minipage} \hfill%
  \begin{minipage}[t]{0.5\textwidth}\sf\flushright \textit{#2}\end{minipage}%
  \par%
}

% Create commands for syntax that you will frequently use
\newcommand{\xx}{\mathbf x}

\begin{document}
\begin{titlepage} 

\lectureheader{Prof. Ryan Cotterell}
{}
{\Large Natural Language Processing}{Spring 2021}
	\newcommand{\HRule}{\rule{\linewidth}{0.5mm}} 
	
	\center % Centre everything on the page
	{\Huge Course Assignment}\\
      \quad\newline
	
	{\large\today} \\
	\quad\newline
	%	Author
	%------------------------------------------------
	
	{\Large Karol Borkowski \\ \emph{nethz} Username: kborkowski \\ Student ID: 00000000}\\[0.5cm] 
	\vfill
	{\large \textbf{Collaborators:} \\
	Other student 1 \\
	Other student 2}
	
	\vfill\vfill\vfill 
	By submitting this work, I verify that it is my own. That is, I have written my own solutions to each problem for which I am submitting an answer. I have listed above all others with whom I have discussed these answers.
	
	\vfill 
	
\end{titlepage}



%%%%%%%%%%%%%%%%%
%   Problem 1   %
%%%%%%%%%%%%%%%%%
\part{Course Assignment Episode 1}

\section*{Question 1} 
\begin{enumerate}[label = (\alph*)]
    \item
    
\begin{figure}[h!]
    \centering
    \includegraphics[scale=1.5]{figures/backprop.jpg}
    \caption{The computation graph of f.}
    \label{fig:roller-coaster}
\end{figure}

$a_1 = x_1 \cdot w_{11}^1$
$b_1 = x_2 \cdot w_{21}^1$
$c_1 = x_3 \cdot w_{31}^1$

$d_1=a_1 + b_1 + c_1$

$h_1 = ReLu(d_1)$

$a_2 = x_1 \cdot w_{12}^1$
$b_2 = x_2 \cdot w_{22}^1$
$c_2 = x_3 \cdot w_{32}^1$

$d_2=a_2 + b_2 + c_2$

$h_2 = ReLu(d_2)$

$a_3 = h_1 \cdot w_{11}^2$
$b_3 = h_2 \cdot w_{21}^2$

$d_3=a_3 + b_3$

$y=\sigma(d_3)$

\item 
	\begin{enumerate}[label = (\roman*)]
		\item
		
		\item
		
		\item
		
		\item
		
	\end{enumerate}

\end{enumerate}





\clearpage
\newpage

\begin{table}[h]
        \centering
        \fontsize{10}{10}\selectfont
        \renewcommand{\arraystretch}{1.2} % vertical padding
        \setlength{\tabcolsep}{0.5em} % for the horizontal padding
        \begin{tabular}{l|l|l|l}
        \textbf{number} & \textbf{sample strings} & \textbf{accepted} & \textbf{weight} \\ \hline
        1 & educational is this not &  &  \\
        2 & is this assignment educational &  &  \\
        3 & not educational is not educational &  &  \\
        4 & this assignment is not educational &  &  \\
        5 & is this assignment educational &  &  \\
        6 & this assignment course is educational &  &  \\
        7 & is this assignment not educational &  &  \\
        8 & this assignment not &  &  \\
        9 & this course assignment is not educational &  &  \\
        10 & this course is not not educational &  &  \\
        11 & not educational is this &  &  \\
        12 & course assignment is not educational &  &  \\
        13 & not this assignment is educational &  &  \\
        14 & not not not educational &  &  \\
        14 & is this course assignment not educational &  &  \\
        15 & course assignment is this &  &  \\
        16 & this course is interesting &  &  \\
        17 & this course assignment not educational &  & 
        \end{tabular}
        \caption{Some strings from $\mathcal{Y}_{\geq 2, \leq 6}$}
        \label{tab:wfst_strings}
        \end{table}
        
\begin{figure}[!h]
        \centering
        \includegraphics[width=1.0\textwidth]{./figures/wfst_charts_1.pdf}
        % \includegraphics[width=.45\linewidth]{./fig/C1000.png}
        \caption{Floyd-Warshall algorithm, iteration 0 to 3; left column matrix should contain weights after iteration n; right column matrix should be iteratively filled for backtracking each path}
        \label{fig:wfst_charts_1}
    \end{figure} 
    
    \begin{figure}[!h]
        \centering
        \includegraphics[width=1.0\textwidth]{./figures/wfst_charts_2.pdf}
        % \includegraphics[width=.45\linewidth]{./fig/C1000.png}
        \caption{Floyd-Warshall algorithm, iteration 4 to 7; left column matrix should contain weights after iteration n; right column matrix should be iteratively filled for backtracking each path}
        \label{fig:wfst_charts_2}
    \end{figure} 
\end{document}
